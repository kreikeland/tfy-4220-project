\section{Discussion}
\textit{link motion shown in the trajectories to the theory}
The trajectories for $|\bm{r}| > 6R_E$ might not make much sense as the dipole model of the magnetic field is not as valid for these distances.

The results from considering the conservation of kinetic energy shows that the kinetic energy grows with time when the protons are close to the Earth. 
This indicates that there is a need for a higher precision numerical scheme for this case.
Moreover, the results show no significant sign of increased numerical instability for protons with lower energy. 
This may be due to the higher precision of RK45 as implemented in SciPy compared to the ordinary RK4 method, as described in (\cite{DORMAND198019}).
Thus it seems reasonable to apply RK45 for the task of simulating the trajectories of charged particles in Earth's magnetoshpere. 
\textit{Can it be that the complexity of RK6 is higher than that of RK45, thus giving credit to RK45 as a superior method for this task seeing as it is less complex yet yields competitive results?}

