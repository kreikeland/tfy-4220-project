\section{Methods}
For this analysis, different trajectories are computed for different initial conditions. 
The simulations are run for protons of kinetic energy between 100eV and 10MeV.
The initial velocities of the protons is then given by $\bm{v_0} = [v,0,0]$ where $v$ is given by equation (\ref{eq:v}) with $m=m_p$.
The initial positions of the protons are given as
\begin{align*}
    x_0 &\in [-2R_E, -12R_E], \\
    y_0 &\in [-R_E, R_E], \\
    z_0 &\in [-R_E,R_E].  
\end{align*}
The values for $x_0,\,y_0,\,z_0$ are sampled evenly in the given intervals with a sample size of $n_x=11$, $n_y,\,n_z = 3$. 
Thus, for each level of kinetic energy, $n_x \cdot n_y \cdot n_z = 99$ trajectories are computed.
All protons are simulated as particles travelling from the Sun to the Earth, i.e. $x_0 < 0$ and $v_y, v_z = 0$.
All trajectories are computed for a timespan of 120s. 

The simulation code is written in the Python programming language. 
The equations of motion (\ref{eq:dvdt}) and (\ref{eq:drdt}) are solved using RK45 as implemented in the Python library SciPy (\cite{scipy_ivp}).
Consult the supplementary material (\textit{include?}) for further details.  
