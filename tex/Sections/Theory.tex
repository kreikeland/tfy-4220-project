\section{Theory}

As a magnetic dipole field, the magnetic field of the Earth is given in coordinate-free form (\cite{griffiths_2024}) as 
\begin{equation}
    \bm B ( \bm r) = \frac{\mu_0}{4\pi r^3} [3(\bm m \cdot \hat{\bm r})\hat{\bm r} - \bm m],
    \label{eq:B_dip}
\end{equation}
where $\bm{m}$ is the magnetic dipole moment of the Earth, $\mu_0$ is the vacuum permeability and $\bm{r} = (x,y,z)$ is the position vector. 
Equation (\ref{eq:B_dip}) can be written in cartesian coordinates as 
\begin{equation}
    \bm{B}(\bm{r}) =\frac{\mu_0}{4\pi r^5} 
    \small{\begin{bmatrix}
        3x^2 + r^2 & 3xy & 3xz \\
        3xy & 3y^2 + r^2 & 3yz \\
        3xz & 3yz & 3z^2 - r^2
    \end{bmatrix}} \bm{m}.
    \label{eq:B-field-cart}
\end{equation}
At the magnetic equator ($\bm{r} \perp \bm{m}$), the magnetic field strength is measured to $B_0 \approx 3.07\times 10^{-5}$T (\cite{soni_2021}).
Therefore, the magnitude of the Earths magnetic dipole moment is given by (\ref{eq:B-field-cart}) as 
\begin{equation}
    m = \frac{4\pi}{\mu_0} R_E^3 B_0,
\end{equation}
where $R_E\approx6370$km is the radius of the Earth. 
In this simulation the Earth is placed at the origin of the coordinate system, and the $x$-axis is along the Sun-Earth axis. 
The magnetic dipole moment $\bm{m}$ is rotated $11\degree + 23\degree$ around the $y$-axis.

The relativistic equation of motion for a charged particle of charge $q$ and mass $m$ moving in a magnetic field is given by the Lorentz force
\begin{equation}
    \gamma m \frac{\text{d}\bm{v}}{\text{d}t} = q\bm{v} \times \bm{B}(\bm{r}),
    \label{eq:dvdt}
\end{equation}
where $\bm{v}$ is the velocity vector of the particle, and $\gamma$ is the Lorentz factor given as
\begin{equation}
    \gamma = \frac{1}{\sqrt{1-v^2/c^2}},
\end{equation}
where $c$ is the speed of light. 
For a particle moving only in a magnetic field, $\gamma$ is constant, as it only depends on the squared magnitude of the velocity, which is unchanged by the Lorentz force (\cite{griffiths_2024}). 
The magnitude of the velocity of the particle is obtained from the kinetic energy of the paricle as
\begin{equation}
    v = c \sqrt{1-\left ( \frac{mc^2}{mc^2 + K}\right )^2}
    \label{eq:v}
\end{equation}
where $K$ is the kinetic energy. 
The position of the particle can be found by integrating the position-velocity relation
\begin{equation}
    \frac{\text{d}\bm{r}}{\text{d}t} = \bm{v}.
    \label{eq:drdt}
\end{equation}
Equations (\ref{eq:dvdt}) and (\ref{eq:drdt}) are solved using RK45 (\cite{DORMAND198019}). 

In a uniform magnetic field, the charged particles will gyrate around the magnetic field lines with a frequency of
\begin{equation}
    \omega_g = \frac{q|\bm{B}|}{\gamma m},
    \label{eq:omega-gyr}
\end{equation}
stemming from the cross product in (\ref{eq:dvdt}). 
The dipole field (\ref{eq:B-field-cart}) however, is non-uniform with a non-zero gradient $\nabla |\bm{B}|$. 
This curvature and gradient of the field introduces two additional forms of motion: a periodic bouncing motion along the magnetic field lines, and an azimuthal drift around the axis of the magnetic dipole (\cite{soni_2021}). 
Due to the component of $\bm{v}$ parallell to $\bm{B}$ in (\ref{eq:dvdt}), the particles move along the magnetic field lines.
When the particle moves towards a region of higher $|\bm{B}|$, it reflects. \textit{why?}
Thus the particle oscillates between these mirror points along the magnetic field lines.
%The frequency at which the particle bounces is given (\textit{source?}) as
%\begin{equation}
%    \omega_b = .
%\end{equation}
According to (\cite{soni_2021}), the gradient is responsible for the azimuthal drift motion. \textit{more info here?}
%The frequency of the drift motion is given (\textit{source?}) as
%\begin{equation}
%    \omega_d = .
%\end{equation}

