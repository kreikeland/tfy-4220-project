\section{Results}
Two-dimensional plots of the magnetic field lines are shown in figure \ref{fig:B-field} for both the $xz$-plane at $y=0$ and the $xy$-plane at $z = -2R_E$. 
Observe that the magnetic poles are tilted away from the $z$-axis.

\import{./Figures/BField/}{BField}


Figure \ref{fig:trajectories} (a-h) shows the projection onto the $xy$- and $xz$-plane of 4 different trajectories computed for different initial conditions. 
The red circle indicates the Earth.
In all cases, the less energetic protons cover less distance in the same time span as the more energetic protons. 
The trajectories given in figures \ref{fig:trajectories} (a-c) and (e-g) display drift motion of the protons around the axis of the magnetic dipole.
Furthermore, the trajectories display both bouncing along the field lines and gyration around the field lines.
The gyro-frequency seem to increase as they approach the mirror points, as depicted in figures \ref{fig:traj-b}, \ref{fig:traj-c}, \ref{fig:traj-f} and \ref{fig:traj-g}.
Figures \ref{fig:traj-a} and \ref{fig:traj-e} display a higher frequency for the drift motion than that of figures \ref{fig:traj-b}, \ref{fig:traj-c}, \ref{fig:traj-f} and \ref{fig:traj-g}.
The bouncing frequency of figure \ref{fig:traj-b} and \ref{fig:traj-f} is evidently higher than that of figures \ref{fig:traj-c} and \ref{fig:traj-g}.
Figures \ref{fig:traj-d} and \ref{fig:traj-h} show the trajectory of a proton starting far away from the Earth.
In this case, the proton escapes the magnetic field of the Earth.
The motion displayed in figures \ref{fig:traj-a} and \ref{fig:traj-e} is still confined to the Earth's magnetoshpere.

\import{./Figures/Trajectories/}{trajectories}

To validate the computed trajectories we use the fact that the kinetic energy is conserved for a charged particle only subject to an external magnetic field (\cite{griffiths_2024}). 
Figure \ref{fig:kinetic-energy} shows the deviation of the kinetic energy for 5 different simulations of protons at 100eV, 10keV and 1MeV as a function of time. 
\import{./Figures/Energy/}{energy} % placement of this governs the whitespace between the initiial condition stuff
The plots indicate that the kinetic energy grows with time for the protons with initial position close to the Earth. 
