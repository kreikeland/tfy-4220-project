\section{Conclusion}
The magnetic field of the Earth was modeled as a magnetic dipole tilted about the $y$-axis.
The trajectories of protons with energies of 100eV, 10keV and 100MeV were calculated via the Lorentz force using RK45. 
The trajectories were computed for 99 different initial conditions.
Most of the trajectories display gyration, azimuthal drift and bouncing motions. 
For some particular inital conditions the motion is not confined.
No motion towards the north or south magnetic poles were observed. 
To model the trajectories of charged particles from solar winds as the source of the auroras, one may need a more realistic model of the Earth's magnetic field. 
The simulations suggest that the motion of protons within Earth's magnetosphere is confined to orbital motion around the Earth, in accordance with (\cite{soni_2021}).
Finally, the results suggest that RK45 is sufficient for simulating motion with $|\bm{r}| \in (4R_E, 6R_E)$. 
For $|\bm{r}| < 4R_E$ a more precise numerical scheme may be needed. 